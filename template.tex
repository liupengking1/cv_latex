%%%%%%%%%%%%%%%%%%%%%%%%%%%%%%%%%%%%%%%%%
% Twenty Seconds Resume/CV
% LaTeX Template
% Version 1.0 (14/7/16)
%
% This template has been downloaded from:
% http://www.LaTeXTemplates.com
%
% Original author:
% Carmine Spagnuolo (cspagnuolo@unisa.it) with major modifications by 
% Vel (vel@LaTeXTemplates.com) and Harsh Gadgil
%
% License:
% The MIT License (see included LICENSE file)
%
%%%%%%%%%%%%%%%%%%%%%%%%%%%%%%%%%%%%%%%%%

%----------------------------------------------------------------------------------------
%	PACKAGES AND OTHER DOCUMENT CONFIGURATIONS
%----------------------------------------------------------------------------------------

\documentclass[letterpaper]{twentysecondcv} % a4paper for A4

% Command for printing skill progress bars
\newcommand\skills{ 
~
	\smartdiagram[bubble diagram]{
        \textbf{Linux},
        \textbf{Embedded}\\\textbf{System},
        \textbf{Mobile}\\\textbf{Devices},
        \textbf{Android}\\\textbf{Dev},
        \textbf{Continuous}\\\textbf{Integration},
        \textbf{SDK}\\\textbf{Design},
        \textbf{Scrum}
    }
}

\interests{{Continuous Integration/4.5},{Embedded systems/5},{Networking Protocols/6},{Software Engineering/6}}

%----------------------------------------------------------------------------------------
%	 PERSONAL INFORMATION
%----------------------------------------------------------------------------------------

% If you don't need one or more of the below, just remove the content leaving the command, e.g. \cvnumberphone{}



\cvname{Peng Liu} % Your name
\cvjobtitle{ Software Engineer, \\ Mobile Developer} % Job title/career

\cvlinkedin{https://linkedin.com/in/peng-liu}
\cvnumberphone{+358 41 7086 582} % Phone number
\cvsite{http://www.pengliu.me/} % Personal website
\cvmail{liupengking1@gmail.com} % Email address

%----------------------------------------------------------------------------------------

\begin{document}

\makeprofile % Print the sidebar

%----------------------------------------------------------------------------------------
%	 EDUCATION
%----------------------------------------------------------------------------------------
\section{Education}

\begin{twenty} % Environment for a list with descriptions
	\twentyitem
    	{2011 - 2015}
        {MSc., Communications Engineering}
        {\href{http://www.aalto.fi/en/}{Aalto University}}
        {Helsinki, Finland}
        {GPA: 4.04/5, Completed with Distinction}
	\twentyitem
    	{2007 - 2011}
        {BEng.,  Electronic and Information Engineering}
        {\href{http://www.xtu.edu.cn/english/}{Xiangtan University}}
        {Xiangtan, Hunan, China}
        {GPA: 3.51/4, Ranked 1/76}
	%\twentyitem{<dates>}{<title>}{<organization>}{<location>}{<description>}
\end{twenty}


\section{Research}
\begin{twenty}
	\twentyitem
    	{June 2012 - \\Dec 2012}
        {Research Assistant}
        {\href{https://www.helsinki.fi/en}{University of Helsinki}}
        {}
        {
        {\begin{itemize}
        \item Participated in the Internet of Things project, set up	 frameworks on Linux to evaluate different protocols over Internet of Things.
        \item Participated in the WiBrA project,	 updated	 the	 implementation of routing protocol in Linux kernel and	 user-space, tested in operator netowrkswith mobile platform, which helped me to gain much experience of IPv6 signaling.
    \end{itemize}}
        }
\end{twenty}

%----------------------------------------------------------------------------------------
%	 EXPERIENCE
%----------------------------------------------------------------------------------------

\section{Experience}

\begin{twenty} % Environment for a list with descriptions
	\twentyitem
    	{Jan 2013 - \\Present}
        {Software Engineer \& Mobile Developer}
        {\href{http://www.tuxera.com/}{Tuxera Inc.}}
        {}
        {
        {\begin{itemize}
        \item Worked in a agile team of 8, finished the design and implementation of the DLNA protocol stack for the AllConnect SDK on both Android and iOS platforms, implemented and released the AllConnect App in Google Play Store and iOS store. Now the app has accumulated over five million downloads and was selected as the 2017 Innovation Award Honoree in software and mobile apps section.
        \item Worked independently on debugging and improving the open source DLNA media server on Linux, ported it to both ASUS-WRT and Android platform. Improved binary passed the DLNA media server certification test suite and UPnP test suite.
        \item Finished my master thesis based on the AllConnect project, implemented and integrated cross-platform multimedia streaming technologies, such as DLNA, Chromecast, Fire TV and AirPlay, in commercial mobile application product.
    \end{itemize}}
        }
        
    \twentyitem
   		{Sep 2012 - \\ May 2013}
        {Project Developer}
        {\href{http://www.aalto.fi/en/}{Aalto University}}
        {}
        {
        {\begin{itemize}
        \item Worked in a team of 9 people with various background. The indoor positioning project was funded and supported by Ericsson and Aalto University. The indoor positioning prototype was built using WiFi fingerprint technology and the final product is an Android application, an Android calibration tool and a positioning engine with room-level accuracy.
    \end{itemize}}
        }
        
        \twentyitem
   		{Dec 2011 - \\ Jan 2013}
        {Head of Services}
        {\href{https://cssa.ayy.fi/}{CSSA-Espoo ry}}
        {}
        {
        \begin{itemize}
        \item Worked as the head of services in the Chinese Students and Scholars Association of Espoo to organize school level events.
    \end{itemize}
    	}
     \twentyitem
   		{Nov 2010 - \\ Jan 2011}
        {Research Assistant}
        {\href{http://www.xtu.edu.cn/english/}{Xiangtan University}}
        {}
        {
        \begin{itemize}
        \item Lead a team of 2, participated in the 2011 National NI virtual instrument contest, created a CDMA communication teaching system using NI LabVIEW 2010.
    \end{itemize}
    	}
        
	%\twentyitem{<dates>}{<title>}{<location>}{<description>}
\end{twenty}

\end{document} 
